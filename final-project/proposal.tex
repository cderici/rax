\documentclass[12pt]{article}
\usepackage[margin=1in]{geometry} 
\usepackage{amsmath,amsthm,amssymb}
\usepackage{enumerate}
\usepackage{latexsym}
\usepackage{xparse}

\begin{document}
 
\title{CSCI-P 523 final project proposal\vspace{-2ex}}
\author{Caner Derici and Ryan Scott} 
 
\maketitle

\section{Description}

Our plan is to enrich our existing R5 compiler with tail call optimization,
i.e., the ability to invoke certain function calls at the end of a function without
the need to allocate an extra stack frame. This would permit the compiler to produce
extremely space-efficient code for recursive functions (which are ubiquitous in
functional programming languages like Racket and R5).

\section{Milestones}

We will work on the final project in the following order:

\begin{enumerate}
 \item Apply tail call optimization only to first-order functions which use immediate
       tail recursion, e.g.     
\begin{verbatim}
(define (fib n acc)
  (if (< n 2) 1 (fib (- n 1) (* n acc))))
\end{verbatim}
 \item Apply tail cail optimization to first-order functions that are mutually
       recursive, e.g.,
\begin{verbatim}
(define (odd?  n)
  (if (eq? n 0) #f (even? (- n 1))))
(define (even? n)
  (if (eq? n 0) #t (odd?  (- n 1))))
\end{verbatim}
 \item Apply tail cail optimization to all tail cails, even those involving
       higher-order functions. For example:
\begin{verbatim}
(define (foo fn)
  (fn 5 1))
(foo fib)
\end{verbatim}

       Implementing this correctly may involve other static analysis techniques
       (e.g., decomposition of the source code into basic blocks).

\end{enumerate}

We are confident in our ability to complete the first two milestones before the final
deadline. The last milestone is less certain, so we will shoot for it as an ideal
goal once the other two milestones have been finished.

\end{document}
