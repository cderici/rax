%%%%%%%%%%%%%%%%%%%%%%%%%%%%%%%%%%%%%%%%%
% Beamer Presentation
% LaTeX Template
% Version 1.0 (10/11/12)
%
% This template has been downloaded from:
% http://www.LaTeXTemplates.com
%
% License:
% CC BY-NC-SA 3.0 (http://creativecommons.org/licenses/by-nc-sa/3.0/)
%
%%%%%%%%%%%%%%%%%%%%%%%%%%%%%%%%%%%%%%%%%

%----------------------------------------------------------------------------------------
%	PACKAGES AND THEMES
%----------------------------------------------------------------------------------------

\documentclass[usenames,dvipsnames]{beamer}

\mode<presentation> {

% The Beamer class comes with a number of default slide themes
% which change the colors and layouts of slides. Below this is a list
% of all the themes, uncomment each in turn to see what they look like.

%\usetheme{default}
%\usetheme{AnnArbor}
%\usetheme{Antibes}
%\usetheme{Bergen}
%\usetheme{Berkeley}
%\usetheme{Berlin}
%\usetheme{Boadilla}
\usetheme{CambridgeUS}
%\usetheme{Copenhagen}
%\usetheme{Darmstadt}
%\usetheme{Dresden}
%\usetheme{Frankfurt}
%\usetheme{Goettingen}
%\usetheme{Hannover}
%\usetheme{Ilmenau}
%\usetheme{JuanLesPins}
%\usetheme{Luebeck}
%\usetheme{Madrid}
%\usetheme{Malmoe}
%\usetheme{Marburg}
%\usetheme{Montpellier}
%\usetheme{PaloAlto}
%\usetheme{Pittsburgh}
%\usetheme{Rochester}
%\usetheme{Singapore}
%\usetheme{Szeged}
%\usetheme{Warsaw}

% As well as themes, the Beamer class has a number of color themes
% for any slide theme. Uncomment each of these in turn to see how it
% changes the colors of your current slide theme.

%\usecolortheme{albatross}
%\usecolortheme{beaver}
%\usecolortheme{beetle}
%\usecolortheme{crane}
%\usecolortheme{dolphin}
%\usecolortheme{dove}
%\usecolortheme{fly}
%\usecolortheme{lily}
%\usecolortheme{orchid}
%\usecolortheme{rose}
%\usecolortheme{seagull}
%\usecolortheme{seahorse}
%\usecolortheme{whale}
%\usecolortheme{wolverine}

%\setbeamertemplate{footline} % To remove the footer line in all slides uncomment this line
%\setbeamertemplate{footline}[page number] % To replace the footer line in all slides with a simple slide count uncomment this line

%\setbeamertemplate{navigation symbols}{} % To remove the navigation symbols from the bottom of all slides uncomment this line
}

\usepackage{graphicx} % Allows including images
\graphicspath{ {images/} }
\usepackage{booktabs} % Allows the use of \toprule, \midrule and \bottomrule in tables
\usepackage{minted}
\usepackage{multicol}
\usepackage{appendixnumberbeamer}

\catcode``=\active
\def`#1`{\texttt{#1}}

%----------------------------------------------------------------------------------------
%	TITLE PAGE
%----------------------------------------------------------------------------------------

\title[Tail-call optimization]{Tail-call optimization in R5} % The short title appears at the bottom of every slide, the full title is only on the title page

\author{Caner Derici and Ryan Scott} % Your name
\date{\today} % Date, can be changed to a custom date

\begin{document}

\begin{frame}
\titlepage % Print the title page as the first slide
\end{frame}

%------------------------------------------------

\begin{frame}
\frametitle{What we aim to do}
\begin{itemize}
\item Currently, R5 allocates a new stack frame for every function call
\item In a functional language, this is terrible, since iteration relies on
      recursive functions (i.e., tons of successive function calls).
\item This can make even simple iteration use up all of one's stack space
      in the blink of an eye.
\end{itemize}
\end{frame}

%------------------------------------------------

\begin{frame}[fragile]
\frametitle{Demo 1}
\begin{minted}{Scheme}
(define (explosion [n : Integer]) : Integer
  (explosion (+ n 0)))

(explosion 0)
\end{minted}
\end{frame}

%------------------------------------------------

\begin{frame}
\frametitle{We can do better}
\begin{itemize}
\item This is awful for a number of reasons:
 \begin{itemize}
  \item Infinite loops should be \textit{infinite}, dagnabbit!
  \item Even programs that aren't infinite might need to recurse many times,
        and hitting the stack frame limit for computations that should terminate
        would be aggravating.
 \end{itemize}
\item We want function calls to be cheaper to allow this sort of programming.
\end{itemize}
\end{frame}

%------------------------------------------------

\begin{frame}[fragile]
\frametitle{Tail calls}
\begin{itemize}
\item In R5, there are certain places---\textit{tail positions}---where one can invoke a function where the
      returned value is given back immediately to the caller.
\item Examples:
\begin{minted}{Scheme}
(if #t
    (foo x)  ; Tail position
    (foo y)) ; Tail position

(let ([x (bar 42)) ; Not tail position
    (baz x))       ; Tail position

(define (quux) : Integer
  ((lambda: () : Integer (quux)))) ; Tail position 
\end{minted}
\end{itemize}
\end{frame}

%------------------------------------------------

\begin{frame}[fragile]
\frametitle{Assembly code for tail calls}
\begin{itemize}
\item This is what our compiler previously generated for \texttt{(explosion)}:
\begin{multicols}{2}
\begin{minted}[fontsize=\tiny]{as}
        .globl explosion
explosion:
        pushq   %rbp
        movq    %rsp, %rbp
        subq    $32, %rsp
        movq    %r13, -16(%rbp)
        movq    %r12, -24(%rbp)
        movq    %rbx, -32(%rbp)

        movq    %rdi, %rbx
        leaq    explosion(%rip), %r13
        movq    free_ptr(%rip), %rsi
        addq    $16, %rsi
        cmpq    fromspace_end(%rip), %rsi
        setl    %al
        movzbq  %al, %rsi
        cmpq    $0, %rsi
        je      then14540
        jmp     end14541
then14540:
        movq    %rbx, %rsi
        addq    $0, %rsi
        movq    %rsi, %rdi
        movq    $16, %rsi
        callq   collect
end14541:
        movq    free_ptr(%rip), %rsi
        addq    $16, free_ptr(%rip)
        movq    $3, 0(%rsi)
        movq    %r13, 8(%rsi)
        movq    $0, %r13
        movq    8(%rsi), %r12
        movq    %rbx, %rdi
        callq   *%r12
        movq    %rax, %rbx
        movq    %rbx, %rax

        movq    -16(%rbp), %r13
        movq    -24(%rbp), %r12
        movq    -32(%rbp), %rbx
        addq    $32, %rsp
        popq    %rbp
        retq
\end{minted}
\end{multicols}

\end{itemize}
\end{frame}

\end{document}