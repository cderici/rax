\documentclass[12pt]{article}
\usepackage[margin=1in]{geometry} 
\usepackage{amsmath,amsthm,amssymb}
\usepackage{enumerate}
\usepackage{latexsym}
\usepackage{xparse}
\usepackage{minted}

\begin{document}

%%%%%%%%%%%%%%%%%%%%%%%%%%%
%%%%%%%% YO CANER %%%%%%%%%
%%%%%%%%%%%%%%%%%%%%%%%%%%%
%
% Here's the report requirements, as listed on canvas.iu.edu:
%
% You final project will include a written report, the source code for your
% compiler (on github), and the tests for your compiler.
%
% The written report should be 3-5 pages for undergraduates and 5-7 pages for
% graduate students. The report should include the following:
%
%   * What problem does your project solve? Or put another way, provide
%     documentation to a hypothetical user of your compiler regarding what your
%     project does for them, without discussing implementation details. Or put
%     yet another way, what are the goals of your project? This part of your
%     report could include example programs, syntax definitions, and definitional
%     interpreters.
%
%   * How does your project solve the problem? Give an overview of your
%     implementation and then discuss the most important details, which should
%     include both the trickiest parts but also the most important parts in
%     terms of having your implementation achieve its stated goals.
%
%   * Provide evidence that your project achieves its goals. This is called an
%     "evaluation". Describe any tests, experiments, or reasoning that you have
%     conducted to evaluate whether your implementation meets your stated goals.
%
%   * Clearly describe which goals have been achieved at the time of turning in
%     your project and which goals were unfinished.
%
%%%%%%%%%%%%%%%%%%%%%%%%%

\title{Tail-call optimization in $R_5$\vspace{-2ex}}
\author{Caner Derici and Ryan Scott} 
 
\maketitle

\begin{abstract}
 TODO: An overview of what TCO, how we implemented it in $R_5$, and what it gained us.
\end{abstract}

\section{Function calls in R5}

\section{Why tail-call optimization?}

\section{Project goals}

\section{Code changes}

\section{Measuring the benefits of TCO}

To instill a sense of appreciation for how much time and space TCO saves when running
compiled programs, we developed two microbenchmarks designed to demonstrate the benefits
of TCO. The first such microbenchmark is the Ackermann function $A$, which is defined
mathematically as follows:

$$
A(m, n) = \begin{cases}
  n+1               & \mbox{if } m = 0 \\
  A(m-1, 1)         & \mbox{if } m > 0 \mbox{ and } n = 0 \\
  A(m-1, A(m, n-1)) & \mbox{if } m > 0 \mbox{ and } n > 0.
\end{cases}
$$

$A$ is of particular interest to computer scientists not only
because of its interesting computability properties, but also because it can lead
to extremely deep levels of recursion, making it useful as a performance benchmark.
Note that while we can define $A$ straightforwardly in $R_5$, it doesn't show off
the benefits of TCO as much as it could due to the call to $A$ in non-tail position
in the third case. Therefore, the program we will benchmark will be converted
to continuation-passing style to leverage more tail calls:

\begin{minted}[fontsize=\small]{Scheme}
(define (ackermann-cps [cont : (Integer -> Integer)]
                       [m : Integer] [n : Integer]) : Integer
  (if (eq? m 0) (cont (+ n 1))
      (if (eq? n 0)
          (ackermann-cps cont (+ m (- 1)) 1)
          (ackermann-cps
            (lambda: ([x : Integer]) : Integer (ackermann-cps cont (+ m (- 1)) x))
            m (+ n (- 1))))))
(define (ackermann [m : Integer] [n : Integer]) : Integer
  (ackermann-cps (lambda: ([x : Integer]) : Integer x) m n))
(ackermann 3 5)
\end{minted}

We compiled this program with two versions of our $R_5$ compiler---
one with TCO enabled, and one without it---and ran the compiled code
using the \verb+/usr/bin/time+ program on Linux.

The results for the compiler with TCO:

\begin{minted}[fontsize=\small]{text}
$ /usr/bin/time -v ./a.out 
253     ...
        Elapsed (wall clock) time (h:mm:ss or m:ss): 0:00.00
        ...
        Maximum resident set size (kbytes): 3252
        ...
        Minor (reclaiming a frame) page faults: 525
        ...
\end{minted}

The results for the compiler without TCO:

\begin{minted}[fontsize=\small]{text}
$ /usr/bin/time -v ./a.out 
253     ...
        Elapsed (wall clock) time (h:mm:ss or m:ss): 0:00.00
        ...
        Maximum resident set size (kbytes): 8560
        ...
        Minor (reclaiming a frame) page faults: 1848
        ...
\end{minted}



\section{Design considerations and future work}

\end{document}
